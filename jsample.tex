
\documentclass[submit]{ipsj}
%\documentclass{ipsj}
%
\usepackage[dvipdfmx]{graphicx}
\usepackage{latexsym}
\usepackage{url}
\usepackage{otf}
\usepackage[top=28truemm,bottom=24truemm,left=15truemm,right=15truemm]{geometry}

\makeatletter
\newcommand\newblock{\hskip .11em\@plus.33em\@minus.07em}
\makeatother

\begin{document}
\title{情報処理学会論文誌ジャーナル論文の準備方法}
\etitle{How to Prepare Your Paper for \\IPSJ Journal Takosu Takosu Takosu Takosu Takosu}

\author{小林 弘人$^{\dag}$}{Hiroto KOBAYASHI$^{\dag}$}{}
\author{入江 豪$^{\dag\dag}$}{Go IRIE$^{\dag\dag}$}{}
\author{古田 諒佑$^{\dag\dag\dag}$}{Ryosuke FURUTA$^{\dag\dag\dag}$}{}
\author{山本 洋太$^{\dag}$}{Yota YAMAMOTO$^{\dag}$}{}
\author{谷口 行信$^{\dag}$}{Yukinobu TANIGUCHI$^{\dag}$}{} 

% TODO: \and も使えるよ 改行したりしそう フィードバックくれ
\ssiiaffiliation{\dag~東京理科大学~〒125--8585~東京都葛飾区新宿6--3--1\\
\dag\dag~NTTコミュニケーション科学基礎研究所~〒243--0198~神奈川県厚木市森の里若宮3--1\\
\dag\dag\dag~東京大学~〒153--8505~東京都目黒区駒場4--6--1}
\ssiieaffiliation{\dag~Tokyo University of Science~6--3--1 Niijuku, Katsushika-ku, Tokyo, 125--8585 Japan\\
\dag\dag~NTT Communication Science Laboratories~3--1 Morinosatowakamiya, Atsugi-shi, Kanagawa, 243--0198 Japan\\
\dag\dag\dag~The University of Tokyo~4--6--1 Komaba, Meguro-ku, Tokyo, 153--8505 Japan}
\ssiiemail{\dag~\url{4620510@ed.tus.ac.jp}}

\begin{abstract}
本稿は,情報処理学会論文誌ジャーナルに投稿する原稿を執筆する際,
および論文採択後に最終原稿を準備する際の注意点等をまとめたものである
大きく分けると,
論文投稿の流れと,\LaTeX と専用のスタイルファイルを用いた場合の論文フォーマットに関する指針,
および論文の内容に関してするべきこと,
するべきでないことをまとめたべからずチェックリストからなる.
本稿自体も \LaTeX と専用のスタイルファイルを用いて執筆されているため,
論文執筆の際に参考になれば幸いである.
\end{abstract}

\begin{jkeyword}
情報処理学会論文誌ジャーナル,\LaTeX,スタイルファイル,べからず集
\end{jkeyword}

\begin{eabstract}
This document is a guide to prepare a draft for submitting to IPSJ
Journal, and the final camera-ready manuscript of a paper to appear in
IPSJ Journal, using {\LaTeX} and special style files.  Since this
document itself is produced with the style files, it will help you to
refer its source file which is distributed with the style files.
\end{eabstract}

\begin{ekeyword}
IPSJ Journal, \LaTeX, style files, ``Dos and Don'ts'' list
\end{ekeyword}

\maketitle

\section{はじめに}
情報処理学会では,基幹論文誌として論文誌ジャーナルの発行を行っている.
現在論文誌ジャーナル編集委員会では,
論文誌ジャーナルの論文掲載時のフォーマットとして
A4縦型2段組を採用している.
また,以前は投稿時と掲載時の形式が異なっていたが,
現在では,
投稿時も掲載時と同様のA4縦型2段組で受け付けることにした.



本稿では,
まずスタイルファイルを用いた論文のフォーマットに関して述べる.
新たなスタイルファイルでは,
極力特別なコマンドは使わずに,標準的な \LaTeX のスタイルを踏襲している.
論文フォーマットに関しては,\ref{sec:format}~章で
後述する指針に従って頂くが,
そこに規定されていること以外は標準的な\LaTeX のコマンドをそのまま使うことができる.
本稿は,そのスタイルファイルを実際に使っているので,
論文執筆の際に参考にされたい.

また,論文誌ジャーナル編集委員会では,論文の執筆する際に,
著者がするべきこと,するべきでないことを「べからず集」としてまとめた.
本稿の後半に,論文の内容に関する指針になるように,
「べからず集」の内容をチェックリストとしてつけているので,
投稿する前の内容のチェックに利用されたい.




%2
\section{投稿の流れ}


%2.1
\subsection{準備}


情報処理学会論文誌ジャーナルの \LaTeX スタイルファイルを含む
論文執筆キットは
からダウンロードすることができる.論文執筆キットは以下のファイルを含んでいる.


Microsoft Wordに関しては,投稿されたフォーマットを基に,
業者が \LaTeX に変換して組版を行うので,
あくまでも参考としてしか使わないことを承知して頂きたい.



%2.2
\subsection{最終原稿の作成と投稿}


なお,情報処理学会論文誌ジャーナルでは,
論文の著者が査読者の名前を知ることがないシングルブラインドの査読を取り入れている.




%2.3
\subsection{最終原稿の作成とファイルの送付}

投稿した論文の採録が決定したら,
査読者からのコメントなどにしたがって原稿を修正し,
図表などのレイアウトも最終的なものとする.
なお後の校正の手間を最小にするために,
この段階で記述の誤りなどを完全に除去するように綿密にチェックして頂きたい.



ファイルの送信方法などについては,
採録通知とともに学会事務局から送られる指示に従う.




%2.4
\subsection{著者校正・組版・出版}


学会では用語や用字を一定の基準(常用漢字および
「現代仮名遣い」等)に従って修正することがある.
また \LaTeX の実行環境の差異などによって著者が作成した最終PDFと
実際の組版結果が微妙に異なることがある.
これらの修正や差異が問題ないかを最終的に確認するために,
著者にPDFファイルが送られるので,
もし問題があれば朱書によって指摘して送信する.
なお{\bf この段階での記述誤りの修正は原則として認められない}ので,
原稿送信時に細心の注意を払っていただきたい.


その後,著者の校正に基づき最終的な組版を行ない,
オンライン出版する.




%3
\section{論文フォーマットの指針}
\label{sec:format}

以下,
情報処理学会論文誌ジャーナル用スタイルファイルを用いた論文フォーマットの
指針について述べるので,
これに従って原稿を用意頂きたい.\LaTeX を用いた
一般的な文章作成技術については,\cite{okumura} 等を参考にされたい.


% TODO: 謝辞がある場合
% \begin{acknowledgment}
% \end{acknowledgment}

\bibliographystyle{junsrt}
\bibliography{references}



% TODO: 付録がある場合
% \appendix



\end{document}
